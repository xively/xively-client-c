\documentclass{beamer}

\mode<presentation>
{
  \usetheme{default}      % or try Darmstadt, Madrid, Warsaw, ...
  \usecolortheme{default} % or try albatross, beaver, crane, ...
  \usefonttheme{default}  % or try serif, structurebold, ...
  \setbeamertemplate{navigation symbols}{}
  \setbeamertemplate{caption}[numbered]
} 

\usepackage[english]{babel}
\usepackage[utf8x]{inputenc}
\usepackage{wasysym}

\title{Connected Embedded Systems}
\subtitle{A Quick Guide to the Three Key Enablers}
\author{Ilya Dmitrichenko}
\date{\today}

\begin{document}

\begin{frame}
        \titlepage
\end{frame}

\begin{frame}{Brief Outline}
        \tableofcontents
\end{frame}

\begin{frame}{Overview (recap)}

\begin{table}\centering\begin{tabular}{l|r|r|r}
Device type & RAM & Flash & CPU clock \\\hline
Small & 16-96 kB & 32-128 kB & 12-150 MHz \\
Large & 128MB-2GB & 32MB-512MB & 400MHz-1.2GHz \\
\end{tabular}\end{table}

\vskip 1cm

\begin{block}{Examples}
\begin{itemize}
  \item[S] coffee maker, door lock, heater, light switch,\\ industrial system component
  \item[L] domestic appliance with touch screen,\\ industrial system controller
  %% may be give current customer example
\end{itemize}
\end{block}

\end{frame}

\section{Hardware Enablers}
\subsection{Device Classification}

\begin{frame}{Classification}

%% This can get pretty blured and hairy, mostly due to the fact that silicon
%% vednors do not always agree between marketing and engineering teams.
%% 
%% Having picked simple terms on the privious slide, we already might want to
%% consider applying familiar classes, e.g. XXL and M\dots
%% 
%% It's best to avoid this and that's why I picked the simplest approach and
%% separate it all into two buckets.

\begin{itemize}
  \item Keep it simple
  \item Only two classes - it's either small or large
  \begin{itemize}
    \item[Small] more like a Microcontroller, aka MCU
    \item[Large] more of an Application System-on-Chip, aka SoC
  \end{itemize}
  \item Hybrid chips are an interesting topic (let's avoid it)
\end{itemize}

\end{frame}

\subsection{Large processor cores}

\begin{frame}{Large processor cores}
\begin{itemize}
  \item General split is around word size (32-bit vs 64-bit)
  \item Embedded systems rarely use 64-bit CPUs
  \item Most common architectures are:
  \begin{itemize}
    \item ARMv7-A (Cortex-A)
    \item ARMv9 and other legacy ARM cores
    \item IBM PowerPC
    \item MIPS
    \item AVR32
    \item Toshiba Supper-H
  \end{itemize}
  \item There are more esoteric architectures that we could dig up\dots
  \item FPGAs and soft-cores are another subject that I'll avoid \frownie
\end{itemize}
\end{frame}

\subsection{Small processor cores}

\begin{frame}{Small processor cores}
\begin{itemize}
  \item Generally they may be 8-, 16-, 24- or 32-bit
  \item ARM dominates the market with ARMv7 and ARMv6 (Cortex-M)
  \item Most vendors publicize which devices use ARM cores
  \item They often add branding of their own, but it's usually clear
  \item Other common architectures:
  \begin{itemize}
    \item Intel 8051
    \item Atmel AVR
    \item Texas Instruments MSP430
    \item Microchip PIC16, PIC24 \& PIC32
    \item Motorolla HCS08 and FreeScale ColdFire
  \end{itemize}
  \item There many other, but those are less common these days
\end{itemize}
\end{frame}

\subsection{Internet and local wireless connectivity}

\begin{frame}{Wireless Evolution}
\begin{itemize}
  \item Free ISM 2.4GHz band worldwide has brought many standards
  \item Most countries allow unlicensed use of various Sub-GHz bands
  \item Sub-GHz radio has better in- and out-door range
  \item New, longer range protocols emerge as UHF TV goes away
  \item A lot of buzz around this, Weightless is an interesting one
  \item Digital RF silicon becomes a commodity
  \item Low-power single-chip solutions are a deal breaker
  \item Energy harvesting is the top tier
  \item Some technologies are more equal then others
  \item Alliances have their own politics and some fail earlier or later 
  \item In the late 80s and early 90s we didn't have Ethernet/IP\dots
\end{itemize}
\end{frame}
\begin{frame}{Wireless Evolution - WTF}
\begin{itemize}
  \item WiMax has already failed
  \item Z-wave has limited success in Europe
  \item WirelessHeart had limited success in industrial applications
  \item ZigBee has numerous issues:
  \begin{itemize}
    \item IEEE 802.15.14 by itself does not guarantee interop
    \item Too many profiles for one to understand
    \item Not all devices on a network are thought to be equal
    \item Non-trivial to bridge with the Internet
    \item Same ZigBee firmware cannot have all profiles
    \item ZigBee/IP solves it to certain extend
    \item ZigBee/IP is only defined for "smart energy" applications
  \end{itemize}
  \item Some customers use proprietary RF procols avoiding compexity
  \item In most cases there are limitation on network size
\end{itemize}
\end{frame}
\begin{frame}{Wireless Evolution - FTW}
\begin{itemize}
  \item Sub-GHz WiFi (802.11ah) is most likely to kill ZigBee
  \item Super-WiFi (802.11af) will likely supersede cellular M2M
  \item Ultra-narrow band is still quite new, we shall see!
  \item Bluetooth LE and NFC are great, but a little less relevant
  \item Mesh or star networks with Bluetooth are non-standard
\end{itemize}
\end{frame}

\section{Tooling Enablers}
\subsection{Compilers and Operating Systems}

\begin{frame}{Background Facts}
\begin{itemize}
  \item LMI quiz: what was Mike's job when dinosaurs where little?
  \item It's only in the last decade when C gained popularity on MCUs
  \item \dots and C$^{++}$ isn't quite there yet for most people \frownie
  \item There used to be proprietary compilers and operating systems
  \item Open-source software and hardware is now most talked about
  \item Linux, GCC and LLVM are leading the game
  \item Many commercial offerings exist around embedded Linux
  \begin{itemize}
    \item GUI toolkits: Qt
    \item Enterprise-grade build systems: Mentor Graphics, Wind River
    \item Cryptographic libraries: Wolf SSL
    \item LLVM-based compilers: Apple, AMD (OpenCL), Qualcomm (DSP), ARM (upcoming)
    \item Toolchain bundles easy to install: Mentor Graphics, Red Hat
  \end{itemize}
  \item Dual-licensing models are quite popular (GPL+commercial)
\end{itemize}
\end{frame}

\begin{frame}{Background Facts (cont.)}
\begin{itemize}
  \item Real-time Linux is hard
  \item Hybrid solutions are common (e.g. Sato)
  \item For small chips, there are still a few commercial vendors
  \item Most live of legacy licensees, only few innovate
  \item Only in the last few years GCC became easier to setup
  \item FreeRTOS is becoming quite popular among silicon vendors
  \item Two TCP/IP stacks are industry standard
  \item Very few other good quality TCP/IP stacks exist
  \item ARM mbed is a great example thought leadership
\end{itemize}
\end{frame}

%% On compilers:
%%
%% If you are using IAR or Rowley or something else of that sort,
%% that means your boss bought a liceses back in a day when GCC
%% was a bit of pain in the ass to get up and running and he was
%% also told that commercial toolchain opimizes a lot better...

\subsection{Languages}

\begin{frame}{Languages - Large devices}
\begin{itemize}
  \item High-level low-footprint runtime is great
  \item Python is already quite popular
  \item Lua and Go are getting great traction
  \item Google is probably giving Java a better push then Oracle does
  \item Java ME has made appearance a while ago\dots
  \item There is now Java Embedded, which is Java 8\dots
  \item The greatest thing is that choices range form shell to Erlang
\end{itemize}
\end{frame}

\begin{frame}{Languages - Small devices}
% Make IoT easy for the masses with JavaScript and Python etc
\begin{itemize}
  \item A number of higher level languages already available
  \begin{itemize}
    \item eLua: \url{https://github.com/elua/elua}
    \item uPy: \url{https://github.com/micropython/micropython}
    \item Espruino: \url{https://github.com/espruino/Espruino}
    \item \href{http://www.oracle.com/technetwork/java/embedded}{Oracle Java Embedded}
  \end{itemize}
  \item Would probably require one to use not the smallest device
  \item Might have a slight performance penalty
  \item Great for defining business logic on top of generic client
  \item IoT blast will happen if development is eased
  \item Other interesting option might be to implement Go (R\&D)
  \item Work with ARM to bind languages to SDK APIs
\end{itemize}
\end{frame}

\section{Cloud Enablers}
\subsection{Background}

\begin{frame}{General Trends}
\begin{itemize}
  \item Cloud becomes a commodity 
  \item Some are able to leverage off-the-shelf platforms (AWS, GCE)
  \item Some find it difficult to grasp the requirements
  \item Some have more difficulties connecting their product
  \item Some are able to build a half-baked prototype
  \item Most Xively customers need professional services
  \item Protocol stack will become a commodity (ARM SensiNode)
  \item Usable open-source protocol stacks will be there in a few years
\end{itemize}
\end{frame}

\subsection{Outlook}

\begin{frame}{Upcoming Xively Messaging Platform}
\begin{itemize}
  \item MQTT might not be the swiss army knife, BUT
  \item The paradigm is easy to understand
  \item Payload-agnostic, efficient compute and bandwidth
  \item Many applications only need simple messaging
  \item More advanced uses would need an RPC\footnote{\href{https://wiki.3amlabs.net/display/XIV/Proposed+RPC+schema+for+RescueX}{wiki.3amlabs.net/display/XIV/Proposed+RPC+schema+for+RescueX}}
\end{itemize}
\end{frame}

\begin{frame}{Future plans and pointers}
\begin{itemize}
  \item Our strategy is to be protocol-agnostic\footnote{\href{http://blog.xively.com/protocol-standardization-aka-iot-red-herring/}{blog.xively.com/protocol-standardization-aka-iot-red-herring/}}
  \item Describing each protocol would take me another day
  \item Here are some links for you to learn more:
  \begin{itemize}
    \item \href{http://youtu.be/4bSr5x5gKvA}{CoAP and LWM2M tutorial}
    \item \href{https://wiki.allseenalliance.org/training}{AllJoyn training}
    \item \href{http://wiki.xmpp.org/web/InternetOfThings}{XMPP}
    \item \href{http://daniel.haxx.se/http2/}{HTTP/2}
  \end{itemize}
  \item Only few of these are currently with \emph{proper} standard bodies
  \begin{itemize}
    \item CoAP - IETF, LWM2M OMA
    \item AllJoyn - Linux Foundation's AllSeen Alliance (still forming)
    \item HTTP/2 - IETF
    \item XMPP - IEEE
    \item MQTT - OASIS (not quite proper)
  \end{itemize}
\end{itemize}
\end{frame}

\end{document}
